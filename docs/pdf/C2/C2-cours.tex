\PassOptionsToPackage{dvipsnames,table}{xcolor}
\documentclass[10pt]{beamer}
\usepackage{Cours}

\begin{document}

\input{\detokenize{/home/fenarius/Travail/Cours/NSITerminale/docs/commun/MacrosCours.tex}}
\setcounter{numchap}{2}
\newcommand{\DB}{\cnum Bases de données et SQL}

% BDD aspect historique
\begin{frame}
	\mframe{\DB}
	\begin{alertblock}{Aspect historique}
		\begin{itemize}
			\item<1-> En 1970, \textcolor{blue}{Edgar Codd} invente le \textit{modèle relationnel} pour les bases de données.
			\item<2-> En 1973, première version du langage \textbf{SQL} (\textbf{S}tructured \textbf{Q}uery \textbf{L}anguage). C'est un langage de requêtes permettant d'interagir avec une base de données.
			\item<3-> En 1979, apparition du premier \textit{S}ystème de \textbf{G}estion de \textbf{B}ase de \textbf{D}onnées ({\sc sgbd}) : Oracle qui fera la fortune de ses trois fondateurs (L. Ellison, E. Oats et B. Miner).
			\item<4-> En 1995 (et 1996), première version d'importants {\sc sgbd} \textcolor{blue}{libres} MySQL (et PostGreSQL).
			\item<5-> En 2020, Le volume mondial de données stockées est estimé à 47 milliards de téra-octets ($47 \times 10^{21}$ octets) et a été multiplié par 20 en 10 ans.
		\end{itemize}
	\end{alertblock}
\end{frame}


% BDD > outils traditionnels
\begin{frame}
	\mframe{\DB}
	\begin{block}{Apport des bases de données}
		Les bases de données corrigent de nombreux défauts des outils traditionnels de gestion des données. En effet :
		\begin{itemize}
			\item<1-> Les accès simultanés par plusieurs programmes aux mêmes données pouvaient générer des conflits.
			\item<2-> Pour lire (ou modifier) les données il fallait savoir comment elles étaient représentées.
			\item<3-> Les utilisateurs devaient s'assurer de  l'intégrité des données avant de les stocker. C'est à dire que c'est l'utilisateur qui était chargé du contrôle de la validité de ses données.
		\end{itemize}
	\end{block}
\end{frame}

% BDD > principes
\begin{frame}
	\mframe{\DB}
	\begin{block}{Principes des bases de données}
		Plusieurs aspects des bases de données viennent corriger les limitations des outils traditionnels :
		\begin{itemize}
			\item<1-> Principe d'\textcolor{red}{unicité} : un enregistrement doit être unique (une donnée qui apparaît plusieurs fois est dite \textit{redondante}). \\
			      \onslide<2->{Ici invervient la notion de \textcolor{red}{clé primaire}, c'est à dire dans une table une identification unique de l'enregistrement.}
			\item<3-> Principe d'\textcolor{red}{intégrite} : le contrôle de la validité des données est effectué par le {\sc sgbd}. \\
			      \onslide<4->{Ici intervient la notion de \textcolor{red}{domaine}, c'est à dire qu'on peut préciser que les valeurs d'un champ doivent être d'un certain types (par exemple entier, flottant, chaine de caractères, ...) et appartenir à un certain ensemble de valeurs  : le domaine.}
			\item<5-> Principe d'\textcolor{red}{indépendance logique} : les utilisateurs accèdent aux données sans se soucier de la façon dont elles sont représentées ou codées dans la base.
		\end{itemize}
	\end{block}
\end{frame}

\begin{frame}
	\mframe{\DB}
	\begin{exampleblock}{Exemple}
		Prenons l'exemple suivant :
		\begin{center}
			{\footnotesize \begin{tabular}{|c|c|c|}
					\hline
					Nom      & Prénom & Naissance \\
					\hline
					Pascal   & Blaise & 1623      \\
					\hline
					Lovelace & Ada    & 1815      \\
					\hline
					Boole    & George & 1815      \\
					\hline
				\end{tabular}} \end{center}
		\begin{itemize}
			\item<1->	Il est certes peu probable (mais pas impossible) que deux personnes portant les mêmes noms et prénoms naissent la même année, afin de respecter le \textcolor{blue}{principe d'unicité}, nous devons adjoindre à chaque enregistrement un champ (par exemple {\tt id}) unique qui servira de clé primaire.
			\item<2-> Les champs {\tt Nom} et {\tt Prénom} sont au format texte, le champ {\tt Naissance} est un entier.
			\item<3-> On peut par exemple préciser les contraintes d'intégrité suivantes : {\tt Nom} doit être non vide, {\tt Naissance} doit être supérieur à 0.
		\end{itemize}
		\onslide<4->{Nous faisons référence à cette base sous le nom \textcolor{OliveGreen}{\tt exemple} dans la suite}
	\end{exampleblock}
\end{frame}


% Schéma relationnel
\begin{frame}
	\mframe{\DB}
	\begin{alertblock}{Schéma relationnel}
		Le schéma relationnel d'une base de données présente les tables de cette base sous la forme de liste ou de tableau. Dans les deux cas, on précise la clé primaire de la table en soulignant l'attribut. On indique aussi parfois le type des attributs.
	\end{alertblock}
	\begin{exampleblock}{Exemple}
		\onslide<2->{Le schéma relationnel de la table {\tt personne} peut s'écrire :}
		\begin{itemize}
			\item<3-> Sous forme de liste : \\
			      \onslide<4->  \textbf{personnes} {\tt (\underline{id} : {\sc int},  {\tt Nom} : {\sc text}, {\tt Prenom} : {\sc text}, {\tt Naissance} : {\sc int})}
			\item<5-> {Sous forme de tableau : \\
			      \begin{tabular}{|lll|}
				      \hline
				      \multicolumn{3}{|c|}{\textbf{personnes}} \\
				      \hline
				      {\tt \underline{id}} & : & {\sc int}     \\
				      \hline
				      {\tt Nom}            & : & {\sc text}    \\
				      \hline
				      {\tt Prenom}         & : & {\sc text}    \\
				      \hline
				      {\tt Naissance}      & : & {\sc int}     \\
				      \hline
			      \end{tabular}}
		\end{itemize}
	\end{exampleblock}
\end{frame}

% Premiers pas en SQL
\begin{frame}
	\mframe{\DB}
	\begin{alertblock}{Premiers pas en SQL}
		\begin{itemize}
			\item<1-> Pour récupérer la totalité des champs d'une table {\tt table} on utilise la syntaxe : \\
			      \onslide<2->{\textcolor{blue}{\sc select} * \textcolor{blue}{\sc from} {\tt table}} \\
			\item<3-> Pour récupérer simplement les champs {\tt champ1, champ2,...} on utilise : \\
			      \onslide<4->{\textcolor{blue}{\sc select} {\tt champ1, champ2,...}  \textcolor{blue}{\sc from} {\tt table}}
		\end{itemize}
	\end{alertblock}
	\begin{exampleblock}{Exemples}
		\begin{itemize}
			\item<5-> {{\sc select} {\tt Nom, Naissance} {\sc from} {\tt exemple}} \\
			      {\footnotesize \begin{tabular}{|c|c|}
				      \hline
				      Pascal   & 1623 \\
				      \hline
				      Lovelace & 1815 \\
				      \hline
				      Boole    & 1815 \\
				      \hline
			      \end{tabular}} \\
		\end{itemize}
	\end{exampleblock}
\end{frame}

% Premiers pas en SQL : clause WHERE
\begin{frame}
	\mframe{\DB}
	\begin{alertblock}{Clause {\sc where}}
		Une instruction \textcolor{blue}{\sc select} peut être suivie d'une clause \textcolor{blue}{\sc where} qui permet de rechercher les enregistrements correspondants à certains conditions. Ces conditions s'expriment à l'aide des opérateurs suivant :
		\begin{itemize}
			\item<2-> Comparaison : {\tt \textcolor{blue}{=}, \textcolor{blue}{<}, \textcolor{blue}{>}, \textcolor{blue}{<=}, \textcolor{blue}{>=},}  {\tt \textcolor{blue}{<>}} (différent)  et \textcolor{blue}{{\sc between}} (entre)
			\item<3-> Logique : \textcolor{blue}{\tt and}, \textcolor{blue}{\tt or} et \textcolor{blue}{\tt not}
			\item<4-> Modèle de chaines de caractères : \textcolor{blue}{\sc like} où \textcolor{blue}{\tt \%} désigne n'importe quel suite de caractères et \textcolor{blue}{\tt \_} un unique caractère.
		\end{itemize}
	\end{alertblock}
	\begin{exampleblock}{Exemples}
		\begin{itemize}
			\item<5-> Pour chercher dans la table les personnes nées après 1789 :\\
			      \onslide<6->{{\sc select * from} {\tt exemple} {\sc where} {\tt naissance > 1789}}
			\item<7-> Pour chercher dans la table les personnes dont la deuxième lettre du nom est un e :\\
			      \onslide<8->{{\sc select * from} {\tt exemple} {\sc where} {\tt nom} {\sc like}  {"\tt \_e\%"}}
		\end{itemize}
	\end{exampleblock}
\end{frame}

% Premiers pas en SQL : Classement des résultats
\begin{frame}
	\mframe{\DB}
	\begin{alertblock}{Clause {\sc order by}}
		Une instruction \textcolor{blue}{\sc select} peut être suivie d'une clause \textcolor{blue}{\sc order by} qui permet de classer les enregistrements selon un ou plusieurs champs. Cette clause est elle même suivie de :
		\begin{itemize}
			\item<2-> \textcolor{blue}{\sc asc} pour indiquer un classement par ordre croissant
			\item<3-> \textcolor{blue}{\sc desc} pour indiquer un classement par ordre décroissant
		\end{itemize}
		\onslide<4->{La valeur par défaut est {\sc asc}}
	\end{alertblock}
	\begin{exampleblock}{Exemples}
		\begin{itemize}
			\item<5-> Pour classer par ordre alphabétique nom puis prénom notre table exemple :\\
			      \onslide<6->{{\sc select * from} {\tt exemple} {\sc order by} {\tt Nom, Prenom} {\sc asc}}
		\end{itemize}
	\end{exampleblock}
\end{frame}

% Premiers pas en SQL : Clause distinct
\begin{frame}
	\mframe{\DB}
	\begin{alertblock}{Clause {\sc distinct} et {\sc limit}}
		\begin{itemize}
			\item<2-> Une instruction \textcolor{blue}{\sc select} peut être \textit{directement} suivie d'une clause \textcolor{blue}{\sc distinct} {\tt champ} qui indique que {\tt champ} ne doit apparaître qu'une fois dans les résultats
			\item<3-> Une instruction \textcolor{blue}{\sc select} peut être suivie d'une clause \textcolor{blue}{\sc limit} qui indique le nombre maximal d'enregistrement à renvoyer. Cette clause est particulièrement utile en relation avec \textcolor{blue}{\sc order by}.
		\end{itemize}
	\end{alertblock}
	\begin{exampleblock}{Exemples}
		\begin{itemize}
			\item<5-> Pour afficher les années de naissance sans répétitions :\\
			      \onslide<6->{{\sc select distinct} {\tt naissance} {\sc from} {\tt exemple}}
			\item<7-> Pour afficher les trois plus jeunes personnes de la table :\\
			      \onslide<6->{{\sc select *  from} {\tt exemple} {\sc order by} {\tt naissance} {\sc desc limit} 3}
		\end{itemize}
	\end{exampleblock}
\end{frame}


% Premiers pas en SQL : agrégation et GROUP BY
\begin{frame}
	\mframe{\DB}
	\begin{alertblock}{Agrégation}
		Le langage {\sc sql} offre des opérateurs appelés \textcolor{blue}{fonction d'agrégation} permettant de calculer une valeur à partir d'un ensemble d'enregistrement :
		\begin{itemize}
			\item<2-> {\sc min} pour obtenir le minimum (d'un champ sur un ensemble d'enregistrement)
			\item<3-> {\sc max} pour obtenir le max
			\item<4-> {\sc sum} pour obtenir la somme
			\item<5-> {\sc avg} pour obtenir le minimum
			\item<6-> {\sc count} pour compter le nombre d'enregistrement
		\end{itemize}
	\end{alertblock}
	\begin{exampleblock}{Exemples}
		\begin{itemize}
			\item<7-> Pour avoir la personne la plus âgée présente dans la table :\\
			      \onslide<8->{{\sc select min}{\tt (Naissance), Nom, Prénom} {\sc from} {\tt exemple}}
		\end{itemize}
	\end{exampleblock}
\end{frame}

\end{document}