\PassOptionsToPackage{dvipsnames,table}{xcolor}
\documentclass[10pt]{beamer}
\usepackage{Cours}

\begin{document}

\input{\detokenize{/home/fenarius/Travail/Cours/NSITerminale/docs/commun/MacrosCours.tex}}
\setcounter{numchap}{3}

\newcommand{\DR}{\cnum Diviser pour régner}

\pythonmode


\begin{frame}
	\mframe{\DR}
	\begin{alertblock}{Principe de la méthode}
		La méthode \textcolor{red}{diviser pour régner} (en anglais \textit{divide and conquer})  est une technique algorithmique qui consiste à :        \begin{enumerate}
            \item<2-> décomposer le problème initial en un ou plusieurs sous problèmes de taille inférieure,
            \item<3-> résoudre chacun des sous problèmes,
            \item<4-> combiner les solutions des sous problèmes pour obtenir la solution au problème initial.
        \end{enumerate}
	\end{alertblock}
    \onslide<5->{
	\begin{exampleblock}{Exemple}
		L'algorithme de \textcolor{blue}{recherche dichotomique} dans un tableau \textbf{trié} déjà rencontré en classe de première est un exemple de la méthode diviser pour régner.
        Le sous problème est alors la recherche dans une liste de taille deux fois plus petite.
	\end{exampleblock}}
\end{frame}

\begin{frame}
	\mframe{\DR}
	\begin{alertblock}{Tri fusion}
		L'algorithme du \textcolor{red}{tri fusion} (en anglais \textit{merge sort})  illustre parfaitement la méthode diviser pour régner, en effet, il consiste à
        \begin{enumerate}
            \item<2-> Décomposer la liste en deux sous listes de longueur égale (à une unité près).
            \item<3-> Trier chacune des sous listes (c'est donc un algorithme récursif)
            \item<4-> Fusionner les parties triées
        \end{enumerate}
	\end{alertblock}
\end{frame}

\begin{frame}
    \mframe{DR}
    \begin{exampleblock}{Exemples}
        Pour illuster la méthode diviser pour régner, on peut aussi citer :
        \begin{itemize}
            \item<2-> La tri rapide (en anglais \textit{quicksort}),
            \item<3-> Le quart de tour d'une image,
            \item<4-> La recherche des deux points les plus proches,
            \item<5-> L'algorithme de multiplication rapide de Karatsuba.
        \end{itemize}
    \end{exampleblock}   
\end{frame}

\end{document}