\PassOptionsToPackage{dvipsnames,table}{xcolor}
\documentclass[10pt]{beamer}
\usepackage{Cours}

\begin{document}

\input{\detokenize{/home/fenarius/Travail/Cours/NSITerminale/docs/commun/MacrosCours.tex}}
\newcommand{\Schema}{\cnum Schéma}
\pythonmode

\newcommand{\Pile}[6]{
    \begin{tabular}{|c|}
        \rnode{#6}{#1} \\
        \hline
        #2 \\
        \hline
        #3 \\
        \hline
        #4 \\
        \hline
        \multicolumn{1}{c}{#5}
    \end{tabular}
}

%Schéma représentatif
\begin{frame}
    \mframe{\Schema}
    \begin{tabular}{c}
    \Pile{}{}{}{}{Départ}{PI} \\
    {\tt <p><em></em></p>} \\
    La pile est initialement vide 
    \end{tabular}
    \begin{tabular}{c}
        \Pile{{}}{}{}{\tt <p>}{Etape 1}{PI} \\
        \textcolor{red}{\tt <p>}{\tt <em></em></p>} \\
        On empile une balise ouvrante
    \end{tabular}
    \begin{tabular}{c}
        \Pile{{}}{}{\tt <em>}{\tt <p>}{Etape 2}{PI} \\
        {\tt <p>}\textcolor{red}{\tt <em>}{\tt </em></p>} \\
        On empile une balise ouvrante
    \end{tabular}
\end{frame}


\begin{frame}
    \mframe{\Schema}
    \begin{tabular}{c}
        \Pile{{}}{}{}{\tt <p>}{Etape 3}{PI} \\
        {\tt <p><em>}\textcolor{red}{\tt </em>}{\tt</p>} \\
        On dépile car la fermante correspond à l'ouvrante empilée.
    \end{tabular}
    \begin{tabular}{c}
        \Pile{{}}{}{}{}{Etape 3}{PI} \\
        {\tt <p><em></em>}\textcolor{red}{\tt </p>} \\
        On dépile car la fermante correspond à l'ouvrante empilée.
    \end{tabular}
\end{frame}

\begin{frame}
    \mframe{\Schema}
    \renewcommand{\arraystretch}{1.4}
    \begin{tabular}{|l|c|c|c|c|}
        \hline
        \textbf{Débit} & 100 kbps & 500 kbps & \textcolor{blue}{10 Mbps} & 100 Mpbs \\
        \hline
        \textbf{Métrique associée} & \textcolor{blue}{1000} & 200 & 10 & 1 \\
        \hline
    \end{tabular}
\end{frame}
    
\end{document}