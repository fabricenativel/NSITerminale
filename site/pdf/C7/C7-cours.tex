\PassOptionsToPackage{dvipsnames,table}{xcolor}
\documentclass[10pt]{beamer}
\usepackage{Cours}

\begin{document}

\input{\detokenize{/home/fenarius/Travail/Cours/NSITerminale/docs/commun/MacrosCours.tex}}
\setcounter{numchap}{7}


\pythonmode
\newcommand{\SOC}{\cnum Systèmes sur puce}


% Définition SoC
\begin{frame}{\SOC}
	\mframe{\SOC}
	\begin{alertblock}{Les sytèmes sur puce}
		\begin{itemize}
			\item<1-> Un ordinateur classique (modèle de Von Neumann) comprend les éléments suivants : un processeur (comprenant une unité arithmétique et logique et une unité de contôle), de la mémoire et des périphériques d'entrées et de sortie.
			\item<2-> Un \textcolor{red}{système sur une puce}, est un circuit intégré réunissant sur le même composant (\textit{puce}) l'ensemble des composants constituant un ordinateur classique.
		\end{itemize}
	\end{alertblock}
	\begin{block}{Remarques}
		\begin{itemize}
			\item<3-> C'est la miniaturisation des composants électroniques qui a permit l'avènement des \textit{SoC}.
			\item<4-> En plus du processeur et de la {\sc ram}, un \textit{SoC} inclut généralement les périphériques réseau (Wifi et Bluetooth) et un circuit graphique ({\sc gpu})
			\item<5-> On trouve des \textit{SoC} notamment dans les téléphones portables, les consoles de jeu portable ou encore les nano ordinateurs comme le Raspberry Pi.
		\end{itemize}
	\end{block}
\end{frame}

% Avantages et inconvénients SoC
\begin{frame}{\SOC}
	\mframe{\SOC}
	\begin{block}{Avantages et inconvénients}
		\begin{itemize}
			\item<1-> Avantages d'un \textit{SoC}:
			\begin{itemize}
				\item<2-> Gain de place
				\item<3-> Consommation réduite d'énergie
				\item<4-> Gain de performance (circuit proches et optimisés)
			\end{itemize}
			\item<5-> Inconvénients d'un \textit{SoC}:
			\begin{itemize}
				\item<2-> Ne peut être réparé, les composants étant intégré si l'un d'entre deux tombe en panne (par exemple le Wifi) on doit changer le \textit{SoC} entier
				\item<3-> N'est pas évolutif, contrairement à  un ordinateur traditionnel où on peut par exemple changer les barrettes de RAM.
			\end{itemize}
		\end{itemize}
	\end{block}
\end{frame}

\end{document}