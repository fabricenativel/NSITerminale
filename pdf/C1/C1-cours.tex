\PassOptionsToPackage{dvipsnames,table}{xcolor}
\documentclass[10pt]{beamer}
\usepackage{Cours}

\begin{document}

\input{\detokenize{/home/fenarius/Travail/Cours/NSITerminale/docs/commun/MacrosCours.tex}}
\setcounter{numchap}{1}
\newcommand{\Recursivite}{\cnum Récursivité}

\pythonmode

% Récursivité : définition et première remarques
\begin{frame}
	\mframe{\Recursivite}
	\begin{alertblock}{Définition}
		\onslide<2->{En informatique, on dit qu'une fonction est \textcolor{red}{récursive},}
		\onslide<3->{lorsque cette fonction fait appel à elle-même.}
	\end{alertblock}
	\onslide<4->{
		\begin{block}{Remarques}
			\begin{itemize}
				\item<5-> Une fonction récursive permet donc, \textit{comme une boucle}, de répéter des instructions. Une même fonction peut donc souvent se programmer de façon \textcolor{blue}{itérative} (avec des boucles) ou de façon \textcolor{blue}{récursive} (en s'appelant elle-même).
				\item<6-> Une fonction récursive doit toujours \textcolor{blue}{contenir une condition d'arrêt}, dans le cas contraire elle s'appelle elle-même à l'infini et le programme ne se termine jamais.
				\item<7-> Les valeurs passées en paramètres lors des appels successifs doivent être différents, sinon la fonction s'exécute à l'identique à chaque appel et donc boucle à l'infini.
			\end{itemize}
		\end{block}}
\end{frame}

% Récursivité : Exemples des puissances
\begin{frame}
	\mframe{\Recursivite}
	\begin{exampleblock}{Exemple : les puissances positives}
		En mathématiques, pour un nombre quelconque $a$ et un entier positif $n$, on définit $a$ puissance $n$ par :\\
		$ a^n = a \times a \times \dots \times a $, et on convient que $a^0=1$
		\begin{itemize}
			\item<2->{Définir une fonction Python {\tt puissance(a,n)} qui retourne $a^n$ en effectuant ce calcul de façon itératif}
			\item<3->{Recopier et compléter : $a^n = \dots \times a^{\dots}$}
			\item<4->{En déduire une version récursive de la fonction calculant les puissances}
		\end{itemize}
	\end{exampleblock}
\end{frame}

% Récursivité : Exemples des puissances
\begin{frame}[fragile]
	\mframe{\Recursivite}
	\begin{exampleblock}{Exemple : les puissances positives}
		\begin{itemize}
			\item<1-> \textcolor{OliveGreen}{Puissance : version itérative}
			      \begin{lstlisting}
def puissance_iteratif(a,n):
    p=1
    for k in range(n):
        p=p*a
    return p
\end{lstlisting}
			\item<2-> $a^n = \textcolor{OliveGreen}{a} \times a^{\textcolor{OliveGreen}{n-1}}$
			\item<3-> \textcolor{OliveGreen}{Puissance : version récursive}
			      \begin{lstlisting}
def puissance_recursif(a,n):
    if n==0:
        return 1
    else:
        return a*puissance_recursif(a,n-1)
\end{lstlisting}
		\end{itemize}
	\end{exampleblock}
\end{frame}


% Récursivité : Exemple du nombre d'occurence récursif - Enoncé
\begin{frame}[fragile]
	\mframe{\Recursivite}
	\begin{exampleblock}{Exemple : une fonction à analyser}
		\begin{lstlisting}
def mystere(elt,liste):
    if liste==[]:
        return 0
    first=liste.pop(0)
    if elt==first:
        return 1+mystere(elt,liste)
    else:
        return mystere(elt,liste)
\end{lstlisting}
		\begin{itemize}
			\item<2-> Que fait la fonction {\tt mystere} ci-dessus ?
			\item<3-> Cette fonction est-elle programmée de façon itérative ? récursive ? Justifier.
			\item<4-> Proposer une version de cette fonction qui ne s'appelle pas elle-même.
		\end{itemize}
	\end{exampleblock}
\end{frame}

% Récursivité : Exemple du nombre d'occurence récursif - Correction
\begin{frame}[fragile]
	\mframe{\Recursivite}
	\begin{exampleblock}{Exemple : une fonction à analyser}
		\begin{itemize}
			\item<2-> \textcolor{OliveGreen}{Cette fonction compte le nombre d'occurence de {\tt elt} dans {\tt liste}}
			\item<3-> \textcolor{OliveGreen}{Elle ne contient pas de boucle, elle n'est donc pas programmé de façon itérative. Par contre c'est une fonction récursive car elle fait appel à elle même.}
			\item<4-> \textcolor{OliveGreen}{Version itérative}
			      \begin{lstlisting}
def occurence(elt,liste):
	occ=0
	for x in liste:
		if x==elt:
			occ=occ+1
	return occ
\end{lstlisting}
		\end{itemize}
	\end{exampleblock}
\end{frame}

% Récursivité : Remarques
\begin{frame}
	\mframe{Recursivite}
	\begin{block}{Remarques importantes}
		\begin{itemize}
			\item<2-> On peut toujours transformer une fonction itérative en son équivalent récursif.
			\item<3-> Certains problèmes (que nous verrons en exercice) ont une solution récursive très lisible et rapide à programmer. La formulation récursive est donc parfois \og plus adaptée \fg à un problème.
			\item<4-> La programmation récursive est parfois gourmande en ressource car les appels récursifs successifs doivent parfois être conservés dans une \textcolor{blue}{pile} dont la taille est limitée.
		\end{itemize}
	\end{block}
\end{frame}

\end{document}