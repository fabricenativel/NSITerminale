\PassOptionsToPackage{dvipsnames,table}{xcolor}
\documentclass[10pt]{beamer}
\usepackage{Cours}

\begin{document}

\input{\detokenize{/home/fenarius/Travail/Cours/NSITerminale/docs/commun/MacrosCours.tex}}
\setcounter{numchap}{13}

\pythonmode

\newcommand{\RT}{\cnum Recherche textuelle}

\pythonmode

\begin{frame}
	\mframe{\RT}
	\begin{block}{Recherche naïve}
		Pour recherche si un motif {\pmc m} se trouve dans une chaîne {\pmc c}, on peut :
		\begin{enumerate}
			\item<1-> parcourir chaque caractère de la chaine {\pmc c}
			\item<2-> si ce caractère correspond au premier caractère du motif {\pmc m},  alors on avance dans le motif tant que les caractères coïncident.
			\item<3-> si on atteint la fin du motif, alors {\pmc m} se trouve dans {\pmc c}. Sinon on passe au caractère suivant de {\pmc c}.
		\end{enumerate}
	\end{block}
	\begin{exampleblock}{Exemple}
		\href{https://boyer-moore.codekodo.net/recherche_naive.php}{Visualisation en ligne du fonctionnement de l'algorithme}
	\end{exampleblock}
\end{frame}

\begin{frame}[fragile]
	\mframe{\RT}
	\begin{block}{Proposition d'implémentation en Python}
		\begin{lstlisting}
	def recherche(motif,chaine):
		lm,lc = len(motif),len(chaine)
		for i in range(lc-lm+1):
			i_motif,i_chaine = 0,i
			while i_motif<lm and chaine[i_chaine] == motif[i_motif]:
				i_motif += 1
				i_chaine += 1
			if i_motif == lm:
				return True
		return False
		\end{lstlisting}
	\end{block}
\end{frame}

\begin{frame}
	\mframe{\RT}
	\begin{block}{Coût de la recherche simple}
		Soient $l_m$ la longueur du motif et $l_c$ celle de la chaine, on vérifie que l'algorithme de recherche simple demande au plus $l_m(l_c-l_m+1)$ comparaisons
	\end{block}
	\begin{exampleblock}{Exemple}
		Combien de comparaisons seront nécessaires si on recherche le motif \pmc{bbbbbbbbba} (neuf fois le caractère \pmc{b} suivi d'un \pmc{a}) dans une chaine contenant un million de \pmc{b} ?
	\end{exampleblock}
\end{frame}

\begin{frame}
	\mframe{\RT}
	\begin{block}{Accélération de la recherche}
		Supposons qu'on recherche le motif \pmc{extra} dans la chaine \pmc{un excellent exemple et un exercice extraordinaire}. La comparaison naïve ci-dessus commence par :\\
		\onslide<2->{
		\begin{tabular}{|c|c|c|c|c|c|c|c|c|c|c|c|}
			\hline
			u              & n &   & e & x & c & e & l & l & e & n & t   \\
			\hline
			\multicolumn{1}{c}{$\updownarrow$} &  \multicolumn{11}{c}{}   \\
			\hline
			e              & x & t & r & a &   &   &   &   &   &   & \\
			\hline
		\end{tabular}}\\
		\onslide<3->{Deux idées vont permettre d'accélérer la recherche :}
		\begin{itemize}
			\item<4-> Commencer par la fin du motif.
			\item<5-> Prétraiter le motif de façon à éviter des comparaisons inutiles.
		\end{itemize}
	\end{block}
\end{frame}

\begin{frame}
	\mframe{\RT}
	\begin{block}{Accélération de la recherche}
		Dans l'exemple ci-dessus cela donne : \\
		\onslide<2->{
		\begin{tabular}{|c|c|c|c|c|c|c|c|c|c|c|c|}
			\hline
			u              & n &   & e & x & c & e & l & l & e & n & t   \\
			\hline
			 \multicolumn{4}{c}{}& \multicolumn{1}{c}{$\updownarrow$} &  \multicolumn{6}{c}{}   \\
			\hline
			e              & x & t & r & a &   &   &   &   &   &   & \\
			\hline
		\end{tabular}}\\
		\onslide<3->{On peut avancer directement de 3 emplacements car le dernier \pmc{x} se trouve à 3 emplacements de la fin du motif.\\}
		\onslide<4->{
		\begin{tabular}{|c|c|c|c|c|c|c|c|c|c|c|c|}
			\hline
			u              & n &   & e & x & c & e & l & l & e & n & t   \\
			\hline
			 \multicolumn{7}{c}{}& \multicolumn{1}{c}{$\updownarrow$} &  \multicolumn{3}{c}{}   \\
			\hline
			& & & e              & x & t & r & a &    &   &   & \\
			\hline
		\end{tabular}\\}
		\onslide<5->{Cette fois, le \pmc{l} ne se trouve pas dans le motif, on peut donc avancer de la longueur du motif.\\}
		\onslide<6->{
			\href{https://boyer-moore.codekodo.net/recherche_boyer.php}{Visualisation en ligne du fonctionnement de l'algorithme accéléré}}
	\end{block}
\end{frame}


\begin{frame}
	\mframe{\RT}
	\begin{block}{Remarques}
		\begin{itemize}
 		\item<1-> L'implémentation, plus délicate que la recherche naïve fait l'objet d'un exercice.
 		\item<2-> L'étude du coût de cet algorithme n'est pas au programme, mais à titre d'exemple, on pourra rechercher le nombre de comparaisons de la recherche du motif \pmc{aaaaaaaaaa} dans un texte  contenant un million de \pmc{b} et comparer avec le cas de la recherche naïve.
		\end{itemize}
	\end{block}
\end{frame}

\end{document}