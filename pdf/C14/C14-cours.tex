\PassOptionsToPackage{dvipsnames,table}{xcolor}
\documentclass[10pt]{beamer}
\usepackage{Cours}

\begin{document}

\input{\detokenize{/home/fenarius/Travail/Cours/NSITerminale/docs/commun/MacrosCours.tex}}
\setcounter{numchap}{14}

\pythonmode

\newcommand{\CD}{\cnum Calculabilité, décidabilité}

\pythonmode

\begin{frame}
	\mframe{\CD}
	\begin{block}{Définitions}
		\begin{itemize}
			\item<1-> Un problème de décision est un problème auquel on peut répondre par \textcolor{blue}{oui} ou \textcolor{blue}{non}\\
			\onslide<2->{\small \textcolor{OliveGreen}{Par exemples : un nombre donné est-il pair ? un graphe contient-il un cycle ?}}
			\item<3-> Un problème est dit  \textcolor{blue}{indécidable} lorsqu'il n'existe pas d'algorithme permettant d'y répondre.\\
			\onslide<4->{\small \textcolor{OliveGreen}{Par exemple, le problème de savoir si un nombre donné est pair n'est pas indécidable. On peut écrire un algorithme qui répond oui ou non à ce problème.}}
			\item<5-> On appelle \textcolor{blue}{problème de l'arrêt} le problème de décision qui consiste à savoir si oui ou non un programme s'arrête.
		\end{itemize}
	\end{block}
	\onslide<6->{
	\begin{alertblock}{Résultat}
		Le problème de l'arrêt est indécidable.
	\end{alertblock}}
\end{frame}

\begin{frame}
	\mframe{\CD}
	\begin{block}{Définition}
			\onslide<1-> Une fonction $f : x \mapsto f(x)$ est dite \textcolor{red}{calculable}, lorsqu'il existe une algorithme permettant de calculer $f(x)$ pour n'importe quelle valeur de $x$.\\
			\onslide<2->{\small \textcolor{OliveGreen}{Par exemple la fonction qui permet de déterminer la longueur du plus court chemin dans un graphe est calculable (voir l'algorithme de Djikstra)}}
	\end{block}
	\onslide<3->{
	\begin{alertblock}{Résultat}
		Le problème de la calculabilité d'une fonction ne dépend pas du langage de programmation.
	\end{alertblock}}
\end{frame}
\end{document}